\section{Objetivos}
El objetivo de esta práctica es desarrollar un sistema que sea capaz de decidir, en el contexto de un videojuego de combate 1 vs. 1 por turnos, cuál de las acciones disponibles es más apropiada en un momento dado valiéndose de Lógica Difusa. 
Dado este entorno, se entiende que en cada instante habrá dos personajes implicados: el controlado por el Sistema de Lógica Difusa, y su adversario (posiblemente controlado por un jugador). Cada uno de los personajes puede realizar una serie de acciones distintas. El objetivo es reducir la vida (representación de la capacidad de un personaje para continuar luchando) del contrario a 0 para ganar el combate.
En concreto, consideraremos el caso en el que nuestro personaje tiene tres posibles acciones: atacar, curarse y defenderse. Además consideramos que nuestra acción se realiza antes que la del adversario. A la hora de tomar las decisiones, nos valdremos de datos de entrada tales como la vida esperada después de realizar una determinada acción, el daño que el personaje puede causar, o la precisión de su ataque.